%      6.   Nomenclature (optional)
%---------------------------
\chapter{Nomenclature}
\label{section:Nomenclature}
The notation below is largely reproduced from \citet{Goodfellow-et-al-2016}.

% From https://github.com/goodfeli/dlbook_notation/blob/master/notation.tex

\vspace{\notationgap}
% Need to use minipage to keep title of table on same page as table
\begin{minipage}{\textwidth}
	% This is a hack to put a little title over the table
	% We cannot use "\section*", etc., they appear in the table of contents.
	% tocdepth does not work on this chapter.
	\centerline{\bf Numbers and Arrays}
	\bgroup
	% The \arraystretch definition here increases the space between rows in the table,
	% so that \displaystyle math has more vertical space.
	\def\arraystretch{1.5}
	\begin{tabular}{cp{3.25in}}
		$\displaystyle a$ & A scalar (integer or real)\\
		$\displaystyle \va$ & A vector\\
		$\displaystyle \mA$ & A matrix\\
	\end{tabular}
	\egroup
	\index{Scalar}
	\index{Vector}
	\index{Matrix}
	\index{Tensor}
\end{minipage}

\vspace{\notationgap}
\begin{minipage}{\textwidth}
	\centerline{\bf Sets and Graphs}
	\bgroup
	\def\arraystretch{1.5}
	\begin{tabular}{cp{3.25in}}
		$\displaystyle \R$ & The set of real numbers \\
		% NOTE: do not use \R^+, because it is ambiguous whether:
		% - It includes 0
		% - It includes only real numbers, or also infinity.
		% We usually do not include infinity, so we may explicitly write
		% [0, \infty) to include 0
		% (0, \infty) to not include 0
		$\displaystyle \{0, 1\}$ & The set containing 0 and 1 \\
		$\displaystyle \{0, 1, \dots, n \}$ & The set of all integers between $0$ and $n$\\
		$\displaystyle [a, b]$ & The real interval including $a$ and $b$\\
		$\displaystyle (a, b]$ & The real interval excluding $a$ but including $b$\\
	\end{tabular}
	\egroup
	\index{Scalar}
	\index{Vector}
	\index{Matrix}
	\index{Tensor}
	\index{Graph}
	\index{Set}
\end{minipage}

\vspace{\notationgap}
\begin{minipage}{\textwidth}
	\centerline{\bf Indexing}
	\bgroup
	\def\arraystretch{1.5}
	\begin{tabular}{cp{3.25in}}
		$\displaystyle \eva_i$ & Element $i$ of vector $\va$, with indexing starting at 1 \\
		$\displaystyle \emA_{i,j}$ & Element $i, j$ of matrix $\mA$ \\
		$\displaystyle \mA_{i, :}$ & Row $i$ of matrix $\mA$ \\
		$\displaystyle \mA_{:, i}$ & Column $i$ of matrix $\mA$ \\
	\end{tabular}
	\egroup
\end{minipage}

\vspace{\notationgap}
\begin{minipage}{\textwidth}
	\centerline{\bf Linear Algebra Operations}
	\bgroup
	\def\arraystretch{1.5}
	\begin{tabular}{cp{3.25in}}
		$\displaystyle \mA^\top$ & Transpose of matrix $\mA$ \\
		$\displaystyle \va^\top$ & Transpose of vector $\va$ \\
		$[\vx_1 \, \ldots \, \vx_P] \in \mathbb{R}^{d \times P}$ & Concatenation where $\vx \in \mathbb{R}^d$
	\end{tabular}
	\egroup
	\index{Transpose}
	\index{Element-wise product|see {Hadamard product}}
	\index{Hadamard product}
	\index{Determinant}
\end{minipage}

\vspace{\notationgap}
\begin{minipage}{\textwidth}
	\centerline{\bf Calculus}
	\bgroup
	\def\arraystretch{1.5}
	\begin{tabular}{cp{3.25in}}
		% NOTE: the [2ex] on the next line adds extra height to that row of the table.
		% Without that command, the fraction on the first line is too tall and collides
		% with the fraction on the second line.
		$\displaystyle\frac{d y} {d x}$ & Derivative of $y$ with respect to $x$\\ [2ex]
		$\displaystyle \frac{\partial y} {\partial x} $ & Partial derivative of $y$ with respect to $x$ \\
	\end{tabular}
	\egroup
	\index{Derivative}
	\index{Integral}
	\index{Jacobian matrix}
	\index{Hessian matrix}
\end{minipage}

\vspace{\notationgap}
\begin{minipage}{\textwidth}
	\centerline{\bf Functions}
	\bgroup
	\def\arraystretch{1.5}
	\begin{tabular}{cp{3.25in}}
		$\displaystyle f(\vx ; \vtheta) $ & A function of $\vx$ parametrized by $\vtheta$. \\
		$\displaystyle \sigma(x)$ & Logistic sigmoid, $\displaystyle \frac{1} {1 + \exp(-x)}$ \\
	\end{tabular}
	\egroup
	\index{Sigmoid}
	\index{Softplus}
	\index{Norm}
\end{minipage}

Sometimes we use a function $f$ whose argument is a scalar but apply
it to a vector, matrix, or tensor: $f(\vx)$, $f(\mX)$, or $f(\tX)$.
This denotes the application of $f$ to the
array element-wise. For example, if $\tC = \sigma(\tX)$, then $\etC_{i,j,k} = \sigma(\etX_{i,j,k})$
for all valid values of $i$, $j$ and $k$.

\clearpage