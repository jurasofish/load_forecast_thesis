%      6.   Nomenclature (optional)
%---------------------------
\chapter{Nomenclature}
\label{section:Nomenclature}
The notation below is largely reproduced from \citet{Goodfellow-et-al-2016}.

% From https://github.com/goodfeli/dlbook_notation/blob/master/notation.tex

\vspace{\notationgap}
% The \arraystretch definition here increases the space between rows in the table,
% so that \displaystyle math has more vertical space.
\def\arraystretch{1.5}
\begin{tabular}{cp{3.25in}}
	$\displaystyle a$ & A scalar (integer or real)\\
	$\displaystyle \va$ & A vector\\
	$\displaystyle \mA$ & A matrix\\
	$\displaystyle \tA$ & A tensor\\
	$\displaystyle \R$ & The set of real numbers \\
	$\displaystyle \eva_i$ & Element $i$ of vector $\va$, with indexing starting at 1 \\
	$\displaystyle \emA_{i,j}$ & Element $i, j$ of matrix $\mA$ \\
	$\displaystyle \mA_{i, :}$ & Row $i$ of matrix $\mA$ \\
	$\displaystyle \mA_{:, i}$ & Column $i$ of matrix $\mA$ \\
	$\displaystyle \mA^\top$ & Transpose of matrix $\mA$ \\
	$\displaystyle \va^\top$ & Transpose of vector $\va$ \\
	$[\vx_1 \, \ldots \, \vx_P] \in \mathbb{R}^{d \times P}$ & Concatenation where $\vx \in \mathbb{R}^d$ \\
	$\frac{d y} {d x}$ & Derivative of $y$ with respect to $x$\\ [2ex]
	$\displaystyle \frac{\partial y} {\partial x} $ & Partial derivative of $y$ with respect to $x$ \\
	$\displaystyle \nabla_\vx y $ & Gradient of $y$ with respect to $\vx$ \\
	$\displaystyle f(\vx ; \vtheta) $ & A function of $\vx$ parametrized by $\vtheta$. \\
	$\displaystyle \sigma(x)$ & Logistic sigmoid, $\displaystyle \frac{1} {1 + \exp(-x)}$ \\
\end{tabular}

Sometimes we use a function $f$ whose argument is a scalar but apply
it to a vector, matrix, or tensor: $f(\vx)$, $f(\mX)$, or $f(\tX)$.
This denotes the application of $f$ to the
array element-wise. For example, if $\tC = \sigma(\tX)$, then $\etC_{i,j,k} = \sigma(\etX_{i,j,k})$
for all valid values of $i$, $j$ and $k$.