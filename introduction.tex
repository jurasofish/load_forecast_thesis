\chapter{Introduction}

Rough Outline
\begin{enumerate}
	\item background
	\item state of the art
	\item problem statement and scope
	\item Analysis and methodology. \\
		  Start from S2S RNN and discuss how this applies in general to NMT.
		  Introduce Transformer architecture as SOTA in NMT.
		  
\end{enumerate}


\section{Background}
Load forecasting, NAC. Plan copy-paste. About one page.

\section{State of the art in load forecasting}
\todo[inline]{I'm not using any of the methods mentioned here - issue?}
\todo[inline]{
	Should each of these subsections be independent? 
	Should they relate to one another?
	Should they simply flow together as natural progressions?
}
\todo[inline]{Do I need all of these sections, or should I talk about RNN and Transformer exclusively?}
Lit review.

\subsection{Classical autoregssive models}
ARIMA, SARIMAX

\subsection{State vector machines}
SVM

\subsection{Gradient boosting}
Extreme (Red Bull EXTREME) Gradient Boosting

\subsection{Clustering and similar day}
k-means is inappropriate for varying sized clusters but when your data's got too many dimensions to visualize no one's gonna stop you.

\subsection{Multilayer perceptron}
Naive, ez

\subsection{Convolutional NN}
"Hey, it works for images why not time series", the short novel by various authors.
something something positional embedding.

\subsection{Recurrent NN}
The real deal. S2S, too. Notably, transformer architecture does not get a mention here.

\section{Problem statement}
predict load blah blah horizon and intervals.

\section{Project scope}
good question - mostly inside the honours propsal if I remember correctly.