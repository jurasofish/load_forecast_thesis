\chapter{Analysis and methodology}

\section{Neural machine translation and time series forecasting}
Neural machine translation is the process of using a neural network to translate a source sentence/phrase from one language, say English, to another language such as Dutch \citep{Cho2014}.
It has been observed that neural network architectures that perform well at NMT also perform well when applied to time series forecasting \hl{(personal observation. is citation required?)}.
This relationship is perhaps intuitive, or perhaps simply coincidental.
\par
A sentence is input to a NMT model by inputting a set of fixed arbitrary dimension vectors, for example a ten word sentence would be represented by ten vectors each with, say, 250 elements, and likewise for the output.
If we wanted to instead supply the model with 48 points from a multivariate time series with three elements at each point, we would replace the set of ten word vectors with a set of 48 vectors, each with three elements.
NMT models are designed to handle variable length sentences, so the change in the number of input vectors is no issue.
In this way, an NMT architecture can be used for time series prediction with minimal modification, opening up the highly competitive (\hl{personal observation}) world of NMT to be taken advantage of.
\par
This section will investigate the performance of several of the best performing NMT architectures when applied to time series forecasting.

\subsection{Investigated models}
The following NMT models will be investigated and their results in time series forecasting compared
\begin{itemize}
	\item sequence to sequence (S2S) \citep{Cho2014a} long short term memory (LSTM) \citep{hochreiter1997long}
	\item S2S LSTM with attention \citep{luong2015effective}
	\item S2S gated recurrent unit (GRU) \citep{Cho2014a} and S2S GRU with attention
	\item Transformer \citep{Vaswani2017}
\end{itemize}

Additionally, I will compare them to the following traditional methods
\begin{itemize}
	\item ARIMA and related methods
	\item support vector regression
	\item I'll probably add a few more as I further work on the state of the art section.
\end{itemize}

\subsection{Evaluation tasks}
The following time series forecasting tasks will be used to evaluate the aforementioned models
\begin{itemize}
	\item forecasting a pure sine wave given it's past values
	\item forecasting a sine wave with normally distributed noise given it's past values
	\item forecasting a signal comprised of several sine waves with normally distributed noise given past values
	\item actually, I might just use the bruny island load data - why not? No need for toy problems then.
\end{itemize}


\section{Transformer}
\hl{from my preliminary work I'm confident that the transformer architecture will be the best, or at least equal best but likely with less expensive training. This is consistent with \protect\cite{Song2017} and \protect\cite{Vaswani2017}}\\
\par
The transformer is a neural network architecture that is currently the state of the art in NMT \citep{Vaswani2017}.
This architecture will be discussed in detail. 

\subsection{Overview}
overview of the transformer architecture.
\begin{itemize}
	\item encoder and decoder
	\item input embedding
	\item positional encoding
	\item multi-head attention
	\item residual connections
	\item feed forward
	\item layer normalization
	\item 
\end{itemize}

\subsection{Input embedding}
convolutional embedding as per \citep{Song2017} (a paper using the transformer architecture to perform classification based on time series).

\subsection{Positional encoding}
Every input vector, $\vb_i$, has a vector added to it, $\vp_i$, depending on its position, $i$, in the input.
$\vp_i$ is trainable - it is drawn from $\vtheta$.

\subsection{multi-head attention}
I need to research this further before writing about it.

\subsection{residual connections}

\subsection{Feedforward}

\subsection{Normalization}

\subsection{etc.}

\subsection{Training}
Discuss how the decoder is used in training and inference modes.
Causality, iterative inference.
	